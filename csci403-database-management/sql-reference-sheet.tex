\documentclass{../cheatsheet}

\course{CSCI 406}
\cheatsheetname{SQL Reference Sheet}

\begin{document}

\begin{multicols*}{2}
    \section{Overview}
    \begin{minted}{postgresql}
        SELECT <attributes>
          FROM <tablenames>
         WHERE <condition>;
    \end{minted}

    \begin{itemize}
        \item \texttt{<attributes>} - the attributes to select
        \item \texttt{<tablenames>} - the table names to select from
        \item \texttt{<condition>} is a boolean expression on the attributes of
            the table

    \end{itemize}

    \section{\texttt{WHERE} Condition}
    \begin{itemize}
        \item \texttt{<>} $\equiv$ not equals

        \item other operators include \texttt{<,>,<=,>=, BETWEEN}.
            (\texttt{BETWEEN} is inclusive.)
            \begin{minted}{postgresql}
                        ... WHERE max_credits BETWEEN 3 AND 6;
            \end{minted}

        \item compound expressions: \texttt{AND, OR, NOT}

        \item Testing for \texttt{NULL}: must use \texttt{IS NULL} or
            \texttt{IS NOT NULL}

        \item \texttt{LIKE} and \texttt{NOT LIKE}
            \begin{minted}{postgresql}
                ... WHERE instructor LIKE 'Paint%';
            \end{minted}

        \item \texttt{IN}
            \begin{minted}{postgresql}
               ... WHERE x IN (1, 2, 3);
            \end{minted}
    \end{itemize}

    \section{Select Statements}
    \subsection{Selecting Expressions on Attributes}
    \begin{minted}{postgresql}
    SELECT 42 / 13 + 12; -- selects 15 (integer math)
    SELECT a || ' ' || b || ' ' || c FROM foo; -- string concatenation
    SELECT substring(a FROM 1 FOR 4) FROM foo; -- first four characters
    \end{minted}

    \subsection{Names and Aliasing}
    \texttt{AS} - used for renaming

    \begin{minted}{postgresql}
    SELECT substring(foo FROM 1 FOR 4) as f, bar as b FROM baz;
    \end{minted}

    \subsection{Schemas}
    \begin{minted}{postgresql}
        -- given "project1" in cpainter and "project1" in public
        SELECT * FROM public.project1; -- selets the public one
    \end{minted}

    \subsection{Misc}
    \begin{minted}{postgresql}
    SELECT count(*) FROM mines_courses_meetings;
    SELECT DISTINCT a1, a2, a3 ...
    \end{minted}

    \section{Joins}
    \begin{minted}{postgresql}
        SELECT * FROM A, B WHERE A.x = B.x;
        SELECT * FROM A JOIN B ON B.x = A.x; -- using join syntax
    \end{minted}

    \section{Order By}
    \begin{minted}{postgresql}
    ... ORDER BY attr DESC/ASC
    \end{minted}

    \section{Table Creation}
    \begin{minted}{postgresql}
        CREATE TABLE [schema_name.]table_name
        (
            attribute1 type1 NOT NULL, -- you can add constraints
            attribute2 type2 PRIMARY KEY,
            attribute3 type3
        )
    \end{minted}
    \begin{minted}{postgresql}
        CREATE TABLE yourid.stuff (
            id serial PRIMARY KEY,
            stuff_id integer REFERENCES yourid.stuff(id), --foreign key inline
            name text NOT NULL,
            PRIMARY KEY (name, age) --implies not null constraints on name, age,

            -- Or can use this to declare foreign key
            FOREIGN KEY (stuff_id) REFERENCES yourid.stuff(id)
        );
    \end{minted}

    \section{Types}
    \begin{itemize}
        \item Integer Types
            \begin{itemize}
                \item \texttt{INTEGER} (32 bit)
                \item \texttt{SMALLINT} (16 bit)
                \item \texttt{BIGINT} (64 bit)
                \item \texttt{SERIAL} auto-incrementing integer
            \end{itemize}

        \item Fixed precision numeric
            \begin{itemize}
                \item \texttt{NUMERIC(w, p)}
                \item \texttt{DECIMAL(w, p)} (an alias for \texttt{NUMERIC})
            \end{itemize}

            Where \texttt{w} = width and \texttt{p} = precision. No more than
            \texttt{w} digits total, \texttt{p} after decimal point.

        \item Floating point
            \begin{itemize}
                \item \texttt{REAL} - 32-bit
                \item \texttt{DOUBLE PRECISION} - 64-bit
            \end{itemize}

        \item Strings
            \begin{itemize}
                \item \texttt{CHAR(n)} - strings of exactly $n$ characters.
                \item \texttt{VARCHAR(n)} - strings up to $n$ characters (space
                    padding not necessary)
                \item (Oracle) \texttt{VARCHAR2(n)} - exactly like \texttt{VARCHAR}
                \item (Postgres) \texttt{TEXT} - USE THIS - essentially infinite
                    width and indexable
            \end{itemize}

        \item Date/Time
            \begin{itemize}
                \item \texttt{DATE} - holds dates. Standard format: `YYYY-MM-DD'.
                \item \texttt{TIME} - holds times. Standard format: `HH:MM:SS' or
                    `HH:MM:SS.nnn' (fractional seconds). By default, this is time
                    without timezone.
                \item \texttt{TIME WITH TIMEZONE} - same as time, but with timezone
                \item \texttt{TIMESTAMP} - time and date
            \end{itemize}

        \item Typecasting
            \begin{minted}{postgresql}
                SELECT CAST('1/2/2016' AS DATE) AS foo;
            \end{minted}

    \end{itemize}

    \section{\texttt{INSERT}, \texttt{DELETE}, \texttt{UPDATE}}
    \begin{minted}{postgresql}
        INSERT INTO table VALUES (v1, v2, ...);
        INSERT INTO table (a1, a2, ..., an) VALUES (v1, v2, ..., vn);
        INSERT INTO table (a1, ..., an) VALUES (v1, ..., vn), (v1, ..., vn)

        -- important to use WHERE condition here
        DELETE FROM table WHERE condition;
        UPDATE table_name SET a1 = v1, a2 = v2, ... WHERE condition;
    \end{minted}

    \newpage
    \section{Aggregate Functions}
    summarize data in a table for some column
    \begin{itemize}
        \item \texttt{COUNT} - counts the non-null entries in a column, or tuples
            \begin{minted}{postgresql}
            SELECT COUNT(*) FROM ...
            \end{minted}

            Can be used with \texttt{DISTINCT} as well.

        \item \texttt{SUM} - adds up the entries in a column

        \item \texttt{MAX/MIN} - the maximum/minimum entry in a column

        \item \texttt{AVG} and others
    \end{itemize}

    \subsection{\texttt{GROUP BY}}
    \begin{minted}{postgresql}
        SELECT a1, a2, ..., f(a3), f(a4), ...
          FROM table1, table2, ... WHERE ...
         GROUP BY a1, a2, ...
         ORDER BY ...
    \end{minted}

    \texttt{a1, a2, ...} must be in the \texttt{GROUP BY}.

    \section{Set Operations in SQL}
    \subsection{Operators}
    \begin{itemize}
        \item \texttt{UNION} - union of two sets of tuples (types and number of
            attributes must be compatible) - all rows in both relations
        \item \texttt{INTERSECTION} - all rows in common
        \item \texttt{EXCEPT} - set difference: all rows in the first relation
            \textbf{not in} the second relation
    \end{itemize}

    \begin{minted}{postgresql}
    SELECT ... UNION SELECT ...;
    SELECT ... INTERSECTION SELECT ...;
    SELECT ... EXCEPT SELECT ...;
    \end{minted}

    \section{Examples}
    \begin{minted}{postgresql}
        SELECT mc.instructor, mc.course_id,
               mef.office, mef.email
          FROM mines_courses as mc, mines_eecs.faculty AS mec
         WHERE mc.instructor = mef.name;
    \end{minted}

    \begin{minted}{postgresql}
        SELECT * FROM foo WHERE bar = 3 ORDER BY alpha, beta, gamma;
    \end{minted}

    \subsection{Grouping}
    \begin{minted}{postgresql}
        SELECT a1, a2, ..., f(a3), f(a4), ...
          FROM ... WHERE ...
         GROUP BY a1, a2, ...;
    \end{minted}

    Result: divides tuples in relation into groups characterized by unique values of
    the group by attributes; aggregates are computed over the groups.

    \begin{minted}{postgresql}
        SELECT COUNT(*), instructor FROM mines_courses
         GROUP BY instructor
         ORDER BY COUNT(*) DESC;
    \end{minted}
\end{multicols*}

\end{document}
