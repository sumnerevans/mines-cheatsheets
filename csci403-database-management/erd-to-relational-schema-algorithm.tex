\documentclass{../cheatsheet}
\newcommand{\ra}{$\rightarrow$\xspace}

\course{CSCI 406}
\cheatsheetname{ERD \ra{} Relational Schema Algorithm}

\begin{document}

\begin{multicols*}{2}
    \section{Regular Entities}
    \begin{itemize}
        \item make a table
        \item simple attributes \ra{} attributes/columns
        \item composite attributes \ra{} just the components become columns
        \item choose a key to be the primary key
        \item uniqueness constraints on other keys
        \item do not include: derived and multivalued
    \end{itemize}

    \section{Weak Entities}
    \begin{itemize}
        \item make a table
        \item all simple attributes \ra{} attributes
        \item components of composite \ra{} attributes
        \item add primary key from owning entity as attribute
        \item make combination of partial and owning entity key \ra{} primary key
        \item foreign key from borrowed key back to key in owning entity
    \end{itemize}

    \section{1:1 Relationships}
    Let $R$ be a 1:1 relationship between entities $S$ and $T$.

    3 choices depending on participation
    \begin{enumerate}[(a)]
        \item $S$ has \textbf{full} participation in $R$, take primary key of $T$
            and add it as attribute to $S$. Make borrowed attribute a foreign key
            back to $T$.
        \item If both $S$ and $T$ fully participate in $R$: $S$ and $T$ can merge
            into one table. (Option (a) is still preferred.)
        \item Not recommended unless there is \textbf{partial} participation on both
            sides. Add a cross-reference table. (Expanded further in step 5.)
    \end{enumerate}

    \section{1:N Relationships}
    Let $R$ be a 1:N relationship between $S$ and $T$

    2 choices depending on participation
    \begin{enumerate}[(a)]
        \item Assuming $T$ has full participation in $R$ (most common case), then
            take primary key from $S$, add to $T$, and make foreign key back to $S$.
        \item (not preferred, unless $T$ has partial participation) cross reference
            table
    \end{enumerate}

    \section{N:M Relationships}
    Let $R$ be an N:M relationship between $S$ and $T$.
    \begin{itemize}
        \item create a new table (called a cross reference table)
        \item add as columns: primary key attribute(s) of $S$, $T$
        \item make each "borrowed" key a foreign key reference back to origin
        \item make all attributes primary key for new table
    \end{itemize}

    \textbf{Example:} using xref table:
    \begin{minted}{postgresql}
    SELECT d.name AS department, d.chair,
           i.name AS instructor, i.office, i.email
      FROM instructor AS i, department AS d,
           instructor_department_xref AS x
     WHERE x.instructor_name = i.name
       AND x.department_name = d.name
       AND d.name = 'Computer Science';
    \end{minted}

    \section{Multi-valued Attributes}
    \begin{itemize}
        \item treat like weak entity
        \item create table containing:
            \begin{itemize}
                \item multivalued attribute
                \item primary key of the entity containing attribute
                \item whole thing is primary key
                \item make column with the primary key of containing entity a
                    foreign key back to the containing table
            \end{itemize}
    \end{itemize}

    \section{N-ary Relationships}
    \begin{itemize}
        \item x-reference table using the primary key attributes of the
            participating tables
    \end{itemize}

    \section{Relational Schema Derived from Above Algorithm}
    \begin{itemize}
        \item \texttt{course (course\_id, title, hours)} (1)
        \item \texttt{instructor (name, office, email)} (1)
        \item \texttt{section (course\_id, section\_id, instructor\_name)} (2, 4)
            \begin{itemize}
                \item FK from \texttt{section.course\_id} to \texttt{course.course\_id}
                \item FK from \texttt{section.instructor\_name} to \texttt{instructor.name}
            \end{itemize}
        \item \texttt{instructor\_department\_xref(instructor\_name,
            department\_name)} (5)
        \item \texttt{section\_meetings(course\_id, section\_id, days, time, room)} (6)
            \begin{itemize}
                \item whole thing is a primary key
                \item foreign key on \texttt{course\_id, section\_id}
            \end{itemize}
    \end{itemize}

    \section{Example of multiple valued primary key}
    \begin{minted}{postgresql}
    CREATE TABLE ...
    columnname type,
    ...
    primary key (col1, col2),
    foreign key (col1) references ...
    \end{minted}


\end{multicols*}

\end{document}
