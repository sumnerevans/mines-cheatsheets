\documentclass{../cheatsheet}

\newcommand{\ra}{$\rightarrow$\xspace}

\course{CSCI 406}
\cheatsheetname{Relational Algebra}

\begin{document}

\begin{multicols*}{2}
    \section{Overview}
    \begin{itemize}
        \item \textbf{tuple}: an ordered and named collection of values

            Ex: \texttt{(`Mehta, Dinesh', `CSCI 406', `ALGORITHMS')}
        \item \textbf{attributes}: formally, the attributes are a set

            Ex: \texttt{(`instructor', `course-id', `title')}
        \item \textbf{domain}: each attribute has associated domain from which
            values come
        \item \textbf{relation schema}:
            \begin{itemize}
                \item A relation schema $R$ is denoted as $R(A_1,
                    A_2,\dots,A_n)$.
                \item Has \textbf{degree} $n$
                \item Attributes $A_i$
                \item Each attribute has associated domain: $Dom(A_i)$
            \end{itemize}

        \item \textbf{relation or relation state}:
            \begin{itemize}
                \item A relation is a set of tuples $r(R)$ conforming to the
                    relation schema: for any tuple $t \in r$ $$t = (v_1, v_2,
                    \dots, v_n) : v_i \in Dom(A_i)$$

                \item In other words: $r(R) \subseteq (dom(A_1) \times dom(A_2)
                    \times \cdots \times dom(A_n))$

                \item \textbf{In actuality} real RDBMS
                    \begin{itemize}
                        \item tables may contain duplicates
                        \item no ordering of tuples
                    \end{itemize}

            \end{itemize}

        \item \textbf{\texttt{NULL}}

            A "value" that can exist in a relational database representing
            \begin{itemize}
                \item a value is unknown
                \item value is irrelevant
                \item value is missing
            \end{itemize}

            \textbf{problems with \texttt{NULL}}
            \begin{itemize}
                \item \texttt{NULL} values cannot be compared
                    $$a = b, a\ is\ NULL, b\ is\ NULL \rightarrow false$$

                    instead we have \texttt{ISNULL} operator
            \end{itemize}

        \item \textbf{constraints}
            \begin{itemize}
                \item on the relation schema
                \item between relation schemas
            \end{itemize}

        \item \textbf{superkey} - A superkey of $R$ is a subset of $R$'s
            attributes such that no relation $r(R)$ may contain tuples with
            exactly the same values for attributes in the superkey.
            \begin{itemize}
                \item any superset of a superkey is also a superkey
                \item every relation (in theory) has at least one superkey - the
                    set of all attributes of the schema
            \end{itemize}

        \item \textbf{Key} - A superkey s.t.\ no attribute can be removed from
            the key without destroying the superkey property: a \textit{minimal}
            superkey

        \item \textbf{Candidate Keys} - All the keys of a schema are called
            \textit{candidate keys}

        \item \textbf{Primary Key} - By convention, we choose one candidate key
            as the \textit{primary key}

            \begin{itemize}
                \item we tend to prefer ``smaller'' keys (e.g. \texttt{crn})
                \item in practice in SQL DB:
                    \begin{itemize}
                        \item primary keys impose uniqueness constraint on data
                        \item other candidate keys $\rightarrow$ impose
                            uniqueness constraint
                    \end{itemize}
            \end{itemize}

        \item \textbf{Referential Integrity/Foreign Key Constraint} Constraint
            on the relationship between relations. A \textit{foreign key} is a
            subset of attributes of a relation schema with the property that its
            values are either \texttt{NULL} or existing in the specified
            attributes of a referenced table.
    \end{itemize}

    \section{Relational Theory}
    \begin{itemize}
        \item each relation is a \textbf{set} of tuples
        \item tuple = set of values associated with named attributes
        \item \textbf{Relational Algebra}
            \begin{itemize}
                \item useful operations that can be applied to relations
                \item resulting algebra
            \end{itemize}
    \end{itemize}

    \section{Unary Operations $-2 + 1$}
    \subsection{Selection}
    \begin{itemize}
        \item Chooses subsets of tuples from a relation according to some
            condition (\texttt{WHERE})

        \item Letting $R$ represent some relation:
            \(\sigma_{\text{condition}}(R)\)

        \item E.g. \( \sigma_{\text{\texttt{course\_id = `CSCI 403'}}}
            (\text{\texttt{mines\_courses}}) \)
        \item Can use \texttt{AND}, \texttt{OR}, \texttt{NOT}, etc.
        \item Properties of $\sigma$
            \begin{itemize}
                \item degree of relation (\# of attributes) is the same as the
                    original
                \item the number of tuples in the result $\leq$ \# tuples in
                    original
                \item commutative \(\sigma_{a}(\sigma_{b}(R)) =
                    \sigma_{b}(\sigma_a(R))\)
                \item any sequence of $\sigma$ operations can be replaced with a
                    single $\sigma$ (using \texttt{AND})
            \end{itemize}
    \end{itemize}

    \subsection{Projection}
    \begin{itemize}
        \item chooses attributes from set of attributes in a relation
        \item parallel operation in SQL: \texttt{SELECT}
        \item Notated:
            \[\pi_{attr_1,attr_2,\dots}(R)\]
        \item Example:
            \[\pi_{instructor, course\_id}(mines\_courses)\]
        \item Properties of $\pi$
            \begin{itemize}
                \item \# of tuples is always $\leq$ \# tuples in original
                \item \textbf{not} commutative: rather
                    \[\pi_{list_1}(\pi_{list_2}(R)) = \pi_{list_1}(R)\]
                    if $list_1 \subset list_2$\\
                    else, ill formed
            \end{itemize}
    \end{itemize}

    \subsection{Renaming - $\rho$}
    \begin{itemize}
        \item parallel operation in SQL: \texttt{AS}
        \item Notation:
            \[\rho_{S(B_1, B_2,\dots)}(R) \]
        \item Example:
            Let table $X$ have attributes $(a, b, c)$.
            \[\rho_{Y(d, e, f)}(X)\]
            \begin{minted}{postgresql}
            SELECT a AS d,
                   b AS e,
                   c AS f
              FROM X as Y;
            \end{minted}
    \end{itemize}

    \subsection{Sequences of Operations}
    \subsubsection{Representations}
    \begin{enumerate}
        \item nesting: example
            \[ \rho_{(name, courseid)}(\pi_{instructor,
            courseid}(\sigma_{department='CS'}(mines\_courses))) \]
            \begin{minted}{postgresql}
        SELECT instructor AS name
               course_id
          FROM mines_courses
         WHERE department = 'CS';
            \end{minted}

        \item sequence of named relations
            \begin{align*}
                R_1 &= \sigma_{department='CS'}(mines\_courses)\\
                R_2 &= \pi_{instructor, course\_id}(R_1)\\
                R_3 &= \rho_{(name, course\_id)}(R_2)
            \end{align*}
    \end{enumerate}

    \section{Binary Operations}
    \subsection{Set Operators}
    \begin{itemize}
        \item $A \cup B$ - union
        \item $A \cap B$ - intersection
        \item $A - B$ - difference
    \end{itemize}

    \subsubsection{Properties of Set Operators}
    \begin{itemize}
        \item Union, intersection are commutative and associative
        \item Set difference is not commutative or associative
    \end{itemize}

    \subsection{Cartesian Product and Joins}
    \begin{itemize}
        \item $A \times B$ - pairs every tuple from $A$ with every tuple from $B$
        \item If $A$ has $m$ attributes, $B$ has $n$, $A \times B$ has $m + n$
            attributes $|A \times B| = |A| \times |B|$.
        \item Typical Usage:
            \[\sigma_{condition}(A \times B) \equiv A \bowtie_{condition} B\]
        \item Example: \(mines\_courses \bowtie_{instructor = name}
            mines\_eecs\_faculty\)
    \end{itemize}

    \subsection{Theta Joins}
    \begin{itemize}
        \item When:
            \begin{itemize}
                \item we have $A \bowtie_{cond} B$
                \item condition is of general form $cond_1 AND cond_2 AND \dots$
                \item each $cond_n$ $A_i \Theta B_j, A_i \in A, B_i \in B$
            \end{itemize}
            Then we call $A \bowtie_{cond}$ a \textit{theta join}: $A
            \bowtie_{\Theta} B$

        \item Further, when $\Theta$ is $=$, then $A \bowtie_\Theta B$ is called
            and \textit{equijoin} which implies that there is some duplicate
            data column.

        \item If $A \bowtie_{cond} B$ is an equijoin and condition equates
            attributes of $A$ with attributes of $B$ of the \textbf{same name},
            then the join is a \textit{natural join}.

            $A * B$ - books notation \\
            $A \bowtie B$ - other people's notation

            Example: both \texttt{mines\_courses} and
            \texttt{mines\_courses\_meetings} have \texttt{crn} field. Then
            $mines\_courses * mines\_courses\_meetings$. (Note, this
            automatically projects away duplicate column(s).)
    \end{itemize}

    \section{Completeness}
    You can demonstrate that: $\sigma, \pi, \cup, -, \times$ is a complete set
    of operators for relational algebra.

    \section{Odds and Ends}
    \begin{itemize}
        \item Division: $\div$ \textapprox inverse of $\times$ (no mirror in
            SQL)
        \item Aggregates and grouping: not part of basic algebra
    \end{itemize}


\end{multicols*}

\end{document}
